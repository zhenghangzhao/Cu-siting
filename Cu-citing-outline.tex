\documentclass[12pt]{article}
\usepackage{lingmacros}
\usepackage{tree-dvips}
\usepackage{enumitem}
\begin{document}

\section*{Cu Exchange and Siting Preference in Zeolites: the Influence of the framework topology}

Zhenghang Zhao\textsuperscript{1} ..... William F. Schneider\textsuperscript{1,*}

{\small
\enumsentence{Department of Chemcial and Biomolecular Engineering, University of Notre Dame, Notre Dame, IN, 46556}
* Corresponding Author: wschnei1@nd.edu 
}                     

\subsection*{Abstract}

Cu-exchange SSZ-13 has been confirmed active for the selective catalytic reduction (SCR) of NO\textsubscript{X} with NH\textsubscript{3} in diesel engine aftertreatment system. The nature of the Cu siting in the SSZ-13 is essential to the redox cycle and the Cu pairability. However, SSZ-13 is a zeolite material with Chabazite topology, which contains only one symmetry-distinct tetrahedral-site and 3-dimensional pore openings. The influence of the zeolite topology in the Cu exchange is still unclear. In this work, we performed Plane-wave-based Density Functional Theory supercell calculations to investigate the Cu exchange and siting preference in 4 different frameworks. We choose FER, AEI and TON to be compared with CHA to address different features of the framework topology including the pore dimensionality, the number of distinct T-site, the double-six-ring building unit, etc. Our results show the similarity and difference between zeolite frameworks

\subsection*{Introduction}
\begin{enumerate}
\item SSZ-13 is on-board catalyst for SCR
\end{enumerate}
Cu-exchange SSZ-13
 
\begin{enumerate}[resume]
\item The Cu exchange and siting have been investigated both experimentally and theoretically
\end{enumerate}

\begin{enumerate}[resume]
\item Other emerging frameworks for SCR, and the uncertainty of the
\end{enumerate}

\begin{enumerate}[resume]
\item Our approach in this paper
\end{enumerate}

\begin{enumerate}[resume]
\item What's new in this work
\end{enumerate}

\subsection*{Computational Methods}
CHA, FER, AEI and TON conventional unit cells were determined from the IZA structure. A 1\times 1\times 1 supercell was constructed for CHA, FER and AEI, while a 1\times 1\times 2 supercell was constructed for TON. For each framework, Si was  substituted by Al to cover all available T-sites. H, Cu and CuOH were placed in all possible locations in each T-site for each framework. A enumeration code was established to enumerate all possible T-site pairs for each framework, and 2 Si were substituted by Al Spin-polarized Density Functional Theory 


\subsection*{Results and Disscussions}
\begin{enumerate}
\item The types of exchange sites in zeolites, Z and Z2 sites
\end{enumerate}
FER has a two-dimensional window structure formed from both 10- and 8-membered-ring channels, and TON has a one-dimensional window structure formed from 10-membered-ring channels. AEI has a three-dimensional window structure formed from 8-membered-ring, but a different stacking of the double 6-ring building unit s from CHA (Figure 1). The consequence of different dimensional window structure is some Cu+ pairs are blocked by small-number-ring walls. We create statistical models using Monte Carlo simulation for Cu\textsuperscript{+} ion pairability. We apply pairing rules using diffusion radius of 9 Å, and window connectivity. The results show when using the diffusion radius as pairing rule, all 4 zeolite frameworks predict the same value of pairable Cu+, because the total number of unique pairs in each framework is nearly the same. When using window connectivity as the pairing rule, AEI and CHA have higher fraction of pairable Cu\textsuperscript{+} compared to FER, while TON predicts significantly less fraction of pairable Cu\textsuperscript{+} than other frameworks do (Figure 2). The reason is that FER and TON are 2- and 1-dimensional framework. When at the reduction-limited regime, active Cu+, inactive Cu+ and Cu\textsuperscript{2+} present in SCR, and our model recovers the fraction of active Cu+. It indicates the dimensionality of zeolite framework controls the fraction of pairable Cu\textsuperscript{+} during SCR oxidation half-cycle. This is consistent with preliminary experimental results that Cu-FER shows slower rates compared to CHA at high O2 pressure limit.


\begin{enumerate}[resume]
\item ZH exchange energies and siting locations
\end{enumerate}
\begin{enumerate}[resume]
\item The ZCuOH exchange energies and siting locations
\end{enumerate}
\begin{enumerate}[resume]
\item The Z2H2 exchange energies and siting locations
\end{enumerate}
\begin{enumerate}[resume]
\item The Z2Cu exchange energies and siting locations
\end{enumerate}
\begin{enumerate}[resume]
\item The Z2Cu and ZCoOH exchange
\end{enumerate}
\begin{enumerate}[resume]
\item The Z2Cu site counting
\end{enumerate}
\begin{enumerate}[resume]
\item The compositional phase diagrams
\end{enumerate}



\subsection*{Conclusions}

In this work, we interrogate and compare Cu+ ion pairability in zeolites through statistical modeling and Monte Carlo simulation, and identify the sensitivity of the pairable Cu+ fraction to the zeolite topology, which promises the potential of rational choice of zeolite topology to tailor the Cu+ pairing during SCR oxidation half-cycle. This approach can be extended to calculate the Cu+ ion pairability of all 237 zeolite topologies for specific SCR applications of interest. 

\subsection*{References}



\end{document}
